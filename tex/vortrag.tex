% !TeX program = lualatex
% Auch „german“ muss angegeben werden, damit die Sprache (z.B. in siunitx)
% auf Deutsch gestellt wird.
\documentclass[german, ngerman]{beamer}
%% =============== Pakete laden =================================
% Typographische Verbesserungen (Mikrotypographie)
\usepackage{microtype}

% Farben ermöglichen
\usepackage{xcolor}
% Einbindung von Bildern ermöglichen (EPS, PNG, JPG, PDF)
\usepackage{graphicx}
% .tex-Dateien mit \includegraphics einbinden
\usepackage{gincltex}
% Bessere Verarbeitung von Dateinamen für \includegraphics etc.
\usepackage{grffile}

%% =============== Paketeinstellungen ===========================
% LaTeX
\renewcommand{\arraystretch}{1.3}
% graphicx
% Standardmäßig „keepaspectratio“ verwenden
% s. https://tex.stackexchange.com/a/91619
\setkeys{Gin}{keepaspectratio}
% hyperref
\hypersetup{unicode}

%% =============== Zusätzliche Einstellungen/Befehle ============
\newcommand*{\email}[1]{\href{mailto:#1}{\texttt{#1}}}


\title[Kurzer Titel]{Ausführlicher Titel}
\subtitle{Untertitel}
% \subject wird von der Vorlage nicht direkt verwendet
%\subject{Thema}
% Autor angeben
\author{Autor}
% \institute wird von der Vorlage nicht direkt verwendet
\institute{Fachschaft Physik}
\date{\today}
\keywords{Münster, Fachschaft Physik}

\begin{document}

% Titelfolie
\begin{frame}[plain]
	\titlepage
\end{frame}

\begin{frame}
	\frametitle{Folien-Überschrift}

	Hier kommt Text!

	\begin{block}{Ein "normaler" Block}
		Inhalt hier.
	\end{block}

	\texttt{itemize} und \texttt{enumerate}:
	\begin{itemize}
		\item Ein Punkt
		\item Noch ein Punkt
		\item Ein dritter Punkt
	\end{itemize}
	\begin{enumerate}
		\item Ein Punkt
		\item Noch ein Punkt
		\item Ein dritter Punkt
	\end{enumerate}
\end{frame}

\begin{frame}
	\frametitle{Ein Alert-Block}

	\begin{alertblock}{Achtung!}
		Hier kommt Rot ins Spiel!
	\end{alertblock}
\end{frame}

\begin{frame}
	\frametitle{Ein Example-Block}

	\begin{exampleblock}{Beispiel}
		Hier kommt Grün ins Spiel!
	\end{exampleblock}
\end{frame}

\begin{frame}
	\begin{block}{}
		\centering
		Vielen Dank für eure Aufmerksamkeit!
	\end{block}

	\begin{block}{}
		\centering
		Habt ihr noch Fragen?
	\end{block}

	\begin{center}
		\includegraphics[width=7cm]{fsphys-logo.pdf}

		\medskip
		\url{https://www.uni-muenster.de/Physik.FSPHYS}
	\end{center}
\end{frame}

\end{document}
