% Author: Simon May
% Date: 2016-10-09

%%%%%%%%%%%%%%%%%%%%%%%%%%%%%%%%%%%%%%%
% Einstellungen zur richtigen Benutzung von wwustyle.sty
\usefonttheme{professionalfonts}

\usepackage{amsmath}
\usepackage{mathtools}

% Überflüssige Warnungen loswerden
\usepackage{silence}
\WarningFilter[spurious]{csquotes}{Using preliminary 'polyglossia' interface}
\ActivateWarningFilters[spurious]

% Schriftarten (Meta) einstellen
\usepackage[quiet]{fontspec}
\usepackage{unicode-math}
\setmainfont{Meta}[Path={fonts/},
	UprightFont=*-Regular,
	UprightFeatures={
		SmallCapsFont={*-Normal-Caps}
	},
	ItalicFont=*-Italic,
	BoldFont=*-Bold,
	BoldFeatures={
		SmallCapsFont={*-Medium-Caps}
	},
	BoldItalicFont=*-Bold,
	BoldItalicFeatures={FakeSlant}
]
\setsansfont{Meta}[Path={fonts/},
	UprightFont=*-Regular,
	UprightFeatures={
		SmallCapsFont={*-Normal-Caps}
	},
	ItalicFont=*-Italic,
	BoldFont=*-Bold,
	BoldFeatures={
		SmallCapsFont={*-Medium-Caps}
	},
	BoldItalicFont=*-Bold,
	BoldItalicFeatures={FakeSlant}
]

% \mathaccent passt wegen unterschiedlicher Schriftarten nicht für lateinische Buchstaben!
\setmathfont{Latin Modern Math}
\setmathfont[range=\mathup/{greek,Greek}, Scale=MatchLowercase]{Free Sans}
\setmathfont[range=\mathit/{greek,Greek}, Scale=MatchLowercase]{Free Sans Oblique}
\setmathfont[range=\mathaccent, Scale=MatchLowercase]{Free Sans}
\setmathfont[range=\mathup/{latin,Latin,num}, Path={fonts/}]{MetaLF-Regular.ttf}
\setmathfont[range=\mathit/{latin,Latin,num}, Path={fonts/}]{MetaLF-Italic.ttf}
% Symbole +, -, ×, ÷, ⋅, ±, %, ! aus Meta nehmen (was ist mit *, :, ∓?)
\setmathfont[range={"0002B,"02212,\times,\div,\cdot,\pm,"00025,"00021}, Path={fonts/}]{MetaLF-Regular.ttf}
\usepackage{polyglossia}
\setmainlanguage{german}

\usepackage[pantone312]{wwustyle}
\usepackage{microtype}
\usepackage{selnolig}
\usepackage{csquotes}

\usepackage{graphicx}
\usepackage{gincltex}
\usepackage{grffile}
\usepackage[useregional]{datetime2}
\usepackage{multirow}
\usepackage{siunitx}

% latex
\renewcommand{\arraystretch}{1.3}
% graphicx
% Standardmäßig „keepaspectratio“ verwenden
% s. https://tex.stackexchange.com/a/91619/51235
\setkeys{Gin}{keepaspectratio}
% hyperref
\hypersetup{unicode}
% siunitx
\sisetup{
	locale=DE,
	quotient-mode=fraction,
	per-mode=fraction,
	fraction-function=\sfrac,
	mode=text
}
% csquotes
\MakeOuterQuote{"}

\institutelogo{\raisebox{-1.75cm}{\includegraphics[width=4cm]{logo.pdf}}}
\institutelogosmall{\raisebox{-1.25cm}{\includegraphics[width=2.6cm]{logo.pdf}}}

%%%%%%%%%%%%%%%%%%%%%%%%%%%%%%%%%%%%%%%
% Zusätzliche Einstellungen/Befehle
\let\strong\textbf
\newcommand{\email}[1]{\href{mailto:#1}{\texttt{#1}}}
\newfontfamily\DejaSans{DejaVu Sans}
