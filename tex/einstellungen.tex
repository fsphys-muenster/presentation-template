% Author: Simon May
% Date: 2017-11-01

%%%%%%%%%%%%%%%%%%%%%%%%%%%%%%%%%%%%%%%
% Einstellungen zur richtigen Benutzung von wwustyle.sty
\usefonttheme{professionalfonts}

\usepackage{amsmath}
\usepackage{mathtools}

% Schriftarten (Meta) einstellen
\usepackage[quiet]{fontspec}
\usepackage{unicode-math}
\setmainfont{Meta}[Path={fonts/},
	UprightFont=*-Regular,
	UprightFeatures={
		SmallCapsFont={*-Normal-Caps}
	},
	ItalicFont=*-Italic,
	BoldFont=*-Bold,
	BoldFeatures={
		SmallCapsFont={*-Medium-Caps}
	},
	BoldItalicFont=*-Bold,
	BoldItalicFeatures={FakeSlant}
]
\setsansfont{Meta}[Path={fonts/},
	UprightFont=*-Regular,
	UprightFeatures={
		SmallCapsFont={*-Normal-Caps}
	},
	ItalicFont=*-Italic,
	BoldFont=*-Bold,
	BoldFeatures={
		SmallCapsFont={*-Medium-Caps}
	},
	BoldItalicFont=*-Bold,
	BoldItalicFeatures={FakeSlant}
]

\setmathfont{Latin Modern Math}
% Sans-Schriftart für griechische Großbuchstaben nutzen
\setmathfont{Latin Modern Sans}[range=up/{Greek}]
\setmathfont{Latin Modern Sans}[range=it/{Greek}]
% Meta für die Bereiche nutzen, für die Glyphen existieren
\setmathfont{MetaLF-Regular.ttf}[Path={fonts/}, range=up/{latin,Latin,num}]
\setmathfont{MetaLF-Italic.ttf}[Path={fonts/}, range=it/{latin,Latin,num}]
% Symbole aus Meta nehmen (enthält leider nicht das Symbol ∓)
\setmathfont{MetaLF-Regular.ttf}[Path={fonts/},
	range={`\+, `\-, `\×, `\÷, `\⋅, `\*, `\/, `\±, `\%, `\‰, `\!, `\?,
		`\., `\,, `\:, `\;}
]
% Am Ende wieder eine echte Mathe-Schriftart laden (range ist egal),
% um Fehler in unicode-math/LuaTeX zu umgehen
% s. https://tex.stackexchange.com/questions/95304
%    https://github.com/wspr/unicode-math/issues/353
\setmathfont{Latin Modern Math}[range=\int]
% Schriftarten \mathrm etc. ebenfalls ändern (für \sin, \exp etc.)
\setmathfontface\mathrm{Meta-Regular.ttf}[Path={fonts/}]
\setmathfontface\mathit{Meta-Italic.ttf}[Path={fonts/}]
\setmathfontface\mathbf{Meta-Bold.ttf}[Path={fonts/}]

\usepackage{polyglossia}
\setmainlanguage{german}

\usepackage[pantone312]{wwustyle}
\usepackage{microtype}
\usepackage{selnolig}
\usepackage{csquotes}

\usepackage{graphicx}
\usepackage{gincltex}
\usepackage{grffile}
\usepackage[useregional]{datetime2}
\usepackage{multirow}
\usepackage{siunitx}

% latex
\renewcommand{\arraystretch}{1.3}
% graphicx
% Standardmäßig „keepaspectratio“ verwenden
% s. https://tex.stackexchange.com/a/91619/51235
\setkeys{Gin}{keepaspectratio}
% hyperref
\hypersetup{unicode}
% siunitx
\sisetup{
	locale=DE,
	quotient-mode=fraction,
	per-mode=fraction,
	fraction-function=\sfrac,
	mode=text
}
% csquotes
\MakeOuterQuote{"}

\institutelogo{\raisebox{-5.75mm}{\includegraphics[width=3.8cm]{fsphys-logo.pdf}}}
\institutelogosmall{\raisebox{-3mm}[0pt][0pt]{\includegraphics[width=2.6cm]{fsphys-logo.pdf}}}

% \framesubtitle aktivieren
\setbeamertemplate{frametitle}{
	\fontsize{14pt}{1pt}
	\selectfont
	\hskip-2.4mm
	\color{maincolor}
	\begin{minipage}[b][20pt]{\textwidth}
		%\textgreater\,
		\insertframetitle
	\end{minipage}
	\ifx\insertframesubtitle\empty%
	\else
	\\
	\vspace{-4pt}
	\fontsize{12pt}{1pt}
	\selectfont
	\hskip-2.4mm
	\color{black}
	\begin{minipage}[b][18pt]{\textwidth}
		%\textgreater\,
		\insertframesubtitle
	\end{minipage}
	\fi%
}

%%%%%%%%%%%%%%%%%%%%%%%%%%%%%%%%%%%%%%%
% Zusätzliche Einstellungen/Befehle
\let\strong\textbf
\newcommand{\email}[1]{\href{mailto:#1}{\texttt{#1}}}
\newfontfamily\DejaSans{DejaVu Sans}
